\documentclass{beamer}
\usepackage{enumerate}
\usepackage[backend=biber]{biblatex}
\usepackage{tikz-cd}

\AtBeginSection[]{
  \begin{frame}
  \vfill
  \centering
  \begin{beamercolorbox}[sep=8pt,center,shadow=true,rounded=true]{title}
    \usebeamerfont{title}\insertsectionhead\par%
  \end{beamercolorbox}
  \vfill
  \end{frame}
}
\setbeamercovered{transparent}


\title{Minimalist Syntax and Mathematics}
\subtitle{A practical approach to Hopf-Algebraic Minimalism in the Classroom}
\author[Fargo]{Andrew Fargo \\Under Profs.~Cynthia~Hansen,~Christy~Hazel}
\institute[Grinnell College]{Grinnell College\\LIN-397-01 IP: New Minimalism \& Alg Frm}

\addbibresource{../project.bib}
\nocite{*}

\date{\today}
\usetheme{Berlin}

\begin{document}

\begin{frame}
  \maketitle
\end{frame}

\begin{frame}
  \frametitle{Table of Contents}
  \tableofcontents
\end{frame}

\section{Introduction and Motivations}

\begin{frame}
  \frametitle{What: Math and Linguistics at Grinnell College}
\end{frame}

\begin{frame}
  \frametitle{So What: Mathematical Literacy}
\end{frame}

\begin{frame}
  \frametitle{Now What: A Course on MCB!}
\end{frame}

\section{Course Syllabus}

\begin{frame}<1>[label=overview]
  \frametitle{Year Overview}
  \begin{enumerate}
  \item<1,4>[Unit 1] Scientific Modeling
    \begin{enumerate}
    \item Types of Scientific Models
    \item Examples of Scientific Models
    \end{enumerate}
  \item<2,4>[Unit 2] Models of Syntax
    \begin{enumerate}
    \item Examples in Generative Linguistics
    \item The Minimalist Program and its Models
    \end{enumerate}
  \item<3,4>[Unit 3] Comparative Review
    \begin{enumerate}
    \item Chomsky's New Minimalism
    \item Constituency (Merge)
    \item X-Linguistic Similarities (Features)
    \item Word Order (Linearization)
    \item Grammaticality (Filtering)
    \item Paraphrases (Possible Derivations)
    \end{enumerate}
  \end{enumerate}
\end{frame}

\begin{frame}
  \frametitle{Unit 1: Scientific Modeling}
  Takeaways:
\begin{enumerate}
\item A mathematical model is a type of scientific model
\item Models are dogmatic. No model is a perfect model.
\item A mathematical model can be evaluated on its expressive power, cost, and empirical soundness.
\item A mathematical model is defined by its objects and axioms.
\end{enumerate}
\end{frame}

\againframe<2>{overview}

\begin{frame}
  \frametitle{Unit 2: Models of Syntax}
  Takeaways:
\begin{enumerate}
\item There exist many models of syntax, particularly in generative linguistics.
\item Chomsky's Heirarchy is a method of evaluating a grammar's expressive power.
\item ``Minimalism'' is not one theory, but a guiding set of principles for several models of syntax.
\end{enumerate}
\end{frame}

\againframe<3>{overview}

\begin{frame}
  \frametitle{Unit 3: Comparative Review}
  Takeaways:
  \begin{enumerate}
  \item ``New minimalism'' is Chomsky's recent revision of the Minimalist Program.
  \item Ed Stabler's ``Computational Minimalism'' and Marcolli, Chomsky, and Berwick's Hopf-Algebraic Minimalism are two appraoches to modelling the minimalist program.
  \item Computational minimalism aligns closer to the ``90s formulation'' of minimalism, whereas Hopf-Algebraic minimalism aligns closer to ``New minimalism.''
  \end{enumerate}
\end{frame}

\againframe<4>{overview}

\begin{frame}
  \frametitle{Optional Unit: Mathematical Prerequisites}
  Course equivalences:
  \begin{itemize}
  \item MAT-322 ``Advanced Topics in Abstract Algebra''
  \end{itemize}

  Covers:
  \begin{itemize}
  \item Elementary Category Theory
  \item Algebras, Bialgebras, Hopf Algebras, and Universal Algebras
  \item Modules over Algebras
  \end{itemize}
\end{frame}

\begin{frame}
  \frametitle{Optional Unit: Minimalist Prerequisites}
  Course equivalences:
  \begin{itemize}
  \item LIN-216 ``Syntax''
  \item LIN-375 ``Advanced Linguistic Analysis'' in Syntax
  \item LIN-295/LIN-395 Special Topic in Syntax
  \end{itemize}

  Covers:
  \begin{itemize}
  \item P\&P Architecture
  \item Governor-Binding Theory
  \item 90s Minimalism
  \end{itemize}
\end{frame}

\section{Formatted Examples}

\begin{frame}
  \frametitle{Example Essay Prompt}
  In three double-spaced pages, answer the following:
  \begin{itemize}
  \item What is the SMT? How does it relate to the miracle creed?
  \item Are you convinced by the concept of the miracle creed? Why or why not?
  \item Give one example for and one example against the existance of the miracle creed in your area of interest. E.g. in Mathematics, one may write about complex-analytic series expansions and G\"odel's incompleteness theorem.
  \end{itemize}
  You may cite \citetitle{chomsky24:_mirac_creed_smt} \cite{chomsky24:_mirac_creed_smt}, \citetitle{chomsky23:_merge_stron_minim_thesis} \cite{chomsky23:_merge_stron_minim_thesis}, or other resources should you find them useful.
\end{frame}

\begin{frame}[fragile]
  \frametitle{Example Worksheet Questions}
  On the \(\mathbb{Q}\)-vector space of workspaces \(\mathcal{WS}:=\mathcal{V}(\mathcal{F}_{\mathcal{SO}_0})\), \citeauthor{marcolli25:mcb} define a product \(\sqcup\), coproduct \(\Delta\), antipode \(S\), and unit and counit \(\epsilon, \eta\). Prove that these operations compatibly form a Hopf Algebra. That is, show that the hexagon
    \[
      \begin{tikzcd}[column sep=tiny]
        &\mathcal{WS}\otimes\mathcal{WS}\arrow[rr,"S\otimes\mathrm{id}"]
        &
        &\mathcal{WS}\otimes\mathcal{WS}\arrow[dr,"\sqcup"]
        &\\
        \mathcal{WS}\arrow[ur,"\Delta"]\arrow[rr,"\epsilon"]\arrow[dr,"\Delta"]
        &
        &\mathbb{Q}\arrow[rr,"\eta"]
        &
        &\mathcal{WS}\\
        &\mathcal{WS}\otimes\mathcal{WS}\arrow[rr,"\mathrm{id}\otimes S"]
        &
        &\mathcal{WS}\otimes\mathcal{WS}\arrow[ur,"\sqcup"]
        &\\
      \end{tikzcd}
    \]
    commutes.
\end{frame}

\begin{frame}
  \frametitle{Example Lecture Notes Excerpt}
\end{frame}

\section{Future Directions}

\begin{frame}
  \frametitle{A course for Grinnell}
\end{frame}

\begin{frame}
  \frametitle{A course for the public}
  
\end{frame}

\begin{frame}
  \frametitle{It's your turn!}
\end{frame}

\begin{frame}[allowframebreaks]
  \frametitle{Bibliography}
  \printbibliography  
\end{frame}


\end{document}
